
%----------------------------------------------------------------------------------------
%	PACKAGES AND OTHER DOCUMENT CONFIGURATIONS
%----------------------------------------------------------------------------------------

\documentclass[
    10pt, % Main document font size
    a4paper, % Paper type, use 'letterpaper' for US Letter paper
    oneside, % One page layout (no page indentation)
%twoside, % Two page layout (page indentation for binding and different headers)
    headinclude,footinclude, % Extra spacing for the header and footer
    BCOR5mm, % Binding correction
]{scrartcl}

\usepackage{graphics, amssymb}
\usepackage{float}
\usepackage{graphicx}
\usepackage{appendix}
\usepackage{adjustbox}
\usepackage{amsmath}
\usepackage{subcaption}
\graphicspath{{figures/}}

\setlength{\parindent}{0pt}

\usepackage[margin=0.5in]{geometry}
\usepackage{cite}


\hyphenation{Fortran hy-phen-ation} % Specify custom hyphenation points in words with dashes where you would like hyphenation to occur, or alternatively, don't put any dashes in a word to stop hyphenation altogether


\title{Measuring DY Asymmetry} % The article title

\author{Oz Amram \& Morris Swartz\textsuperscript{1}} % The article author(s) - author affiliations need to be specified in the AUTHOR AFFILIATIONS block

\date{} % An optional date to appear under the author(s)

%----------------------------------------------------------------------------------------

\begin{document}



\maketitle % Print the title/author/date block

\setcounter{tocdepth}{2} % Set the depth of the table of contents to show sections and subsections only

\tableofcontents % Print the table of contents



\section*{Abstract} % This section will not appear in the table of contents due to the star (\section*)
Placeholder


%----------------------------------------------------------------------------------------
%	AUTHOR AFFILIATIONS
%----------------------------------------------------------------------------------------

{\let\thefootnote\relax\footnotetext{1 \textit{Department of Physics and Astronomy, Johns Hopkins University, }}}



\newpage % Start the article content on the second page, remove this if you have a longer abstract that goes onto the second page


\section{Introduction}

The forward-backward asymmetry, A$_{FB}$, in Drell-Yan processes, $q\bar q\to\ell^+\ell^-$,  arises from the presence of both vector 
and axial vector couplings of gauge bosons to leptons. Prior to data collection, CMS studied its ability to
make a measurement of A$_{FB}$ prior to the beginning of data taking and its ability to serve as an indication of physics beyond the Standard 
Model \cite{Cousins:2005uf}. 
It has also been measured by twice by CMS, once from data collected at 7 TeV
\cite{CMS:AFB_7TeV} and 
once from data collected at 8 TeV \cite{CMS:AFB_8TeV}. 
The start of
data taking at 13 TeV presents the opportunity to measure A$_{FB}$ at higher masses than were possible at 8 TeV,
and in particular, extend the measurement far above the Z-peak. 
Deviations from the Standard Model predictions of A$_{FB}$ could result from the presence of new neutral gauge boson, Z', 
or other new physics scenarios (INSERT REFS HERE). These deviations could be seen at masses significantly lower than the 
mass of the Z', thus this analysis offers a complementary approach in searches for new physics, compared to a traditional 'bump-hunt' approach. \\

A$_{FB}$ has been traditionally defined as follows. First let, $\theta^{*}$ denote the angle between in the incident quark and the final state 
$\ell^-$ in the $\ell^+\ell^-$ cm frame. A$_{FB}$ is then given by:

\begin{equation}
    \label{eq:afb_def}
    A_{FB} = \frac{\sigma_F - \sigma_B}{\sigma_F + \sigma_B}
\end{equation}

where $\sigma_F$ ($\sigma_B$) is the total cross section for forward (backward) events defined by $\theta^* > 0$ ($\theta^* < 0$). 
Rather than directly counting forward and backward events, we adopt a likelihood approach based on the form of the differential cross-section. 
We define $x_1$ and $x_2$ to be the momentum fractions of the incident partons, ordered such that ``Feynman x'', $x_F = x1 - x2$ is positive. 
The invariant mass of the lepton pair, M, is then related to the momentum fractions by $M^2 = x_1 x_2 s$ where $s$ is the square of the center-of-mass (cm) energy. 
The differential cross section for $q\bar q\to\ell^+\ell^-$ production is composed of signal and background parts,

\begin{equation}
    \label{eq:tot_xsec}
    \frac{d^3\sigma}{dx_\mathrm{F}dMdc_*} =\frac{2M}{s\sqrt{x_\mathrm{F}^2+4M^2/s}}\biggl\lbrace\frac{d\sigma}{dc_*}(q\bar q;M^2)\left[D_q(x_1)D_{\bar q}(x_2)+D_q(x_2)D_{\bar q}(x_1)\right] 
    \biggr\rbrace + \frac{d^3\sigma}{dx_\mathrm{F}dMdc_*}(\mathrm{bkg})
\end{equation}

where $c_*\equiv\cos{\theta^*}$ and the tree-level cross section for $q\bar q \to \ell^+\ell^-$ is

\begin{equation}
    \label{eq:qqbar_xsec}
    \frac{d\sigma}{dc_*}(q\bar q;M^2) = \frac{\pi\alpha^2}{2M^2}K(M^2)\biggl\lbrace1+c_*^2+\alpha\left(1-c_*^2\right)
    +\frac{4}{3}\left[2+\alpha\right]A_\mathrm{FB}(M^2)c_*\biggr\rbrace
\end{equation}


where the asymmetry $A_\mathrm{FB}$, normalization parameter $K$, and longitudinal polarization $\alpha$ are functions of $M^2$.  
Note that the asymmetry is characterized by the slope of the linear term in $c_*$.  
A test of this approach was performed by fitting the full NLO angular distribution generated by Powheg to the form given in equation \ref{eq:qqbar_xsec} 
and by comparing the slope asymmetry with the asymmetry determined from counting the forward and backward events.  
Events which are initiated by $q\bar q$ annihilation are considered by themselves and in combination with $qg$ and $\bar q g$ events [the physical case].  
The results are listed in Table~\ref{tab:afb_counting_fitting} for the full sample.  Excellent agreement is observed.  
Note that including $qg$ events which are all NLO increases the longitudinal gauge boson polarization.


\begin{table}[hbt]
    \centering
    \caption{The $q\bar q\to \ell^+\ell^-$ forward-backward asymmetry as determined from a sample of Powheg NLO generated events by counting and by fitting to the function.}
    \label{tab:afb_counting_fitting}
    \vspace{3pt}
    \begin{tabular}{|l|ccc|}\hline
        Sample      & $A_{FB}$ (counting) & $A_{FB}$ (fitting) & $\alpha$ \\ \hline
        All events, $M>300$~GeV  & $0.591\pm0.002$  & $0.591\pm0.002$  & 0.071     \\ 
        $q\bar q$, $M>300$~GeV  & $0.611\pm0.002$  & $0.612\pm0.002$  & 0.044    \\ 
        All events, $M<120$~GeV  & $-0.018\pm0.001$  & $-0.019\pm0.001$  & 0.156     \\ 
        $q\bar q$, $M<120$~GeV  & $-0.014\pm0.001$  & $-0.015\pm0.001$  & 0.004     \\ 
        \hline
    \end{tabular}
\end{table}

In order to reduce the effects of transverse momentum of the incident quarks ($p_t$), we adopt the Colins-Soper (INSERT REF HERE). In this frame,
$\theta^*_{CS}$ is defined as the angle between the negatively charged lepton and the axis that bisects the angle between between the incident quark direction 
and incident anti-quark direction. Approximating the leptons as massless, $cos \theta^*_{CS}$ can be computed from the lab frame variables as:

\begin{equation}
    \label{eq:cs}
    cos \theta^{*}_{CS} = \frac{2(l^+ \bar l^- - l^- \bar l^+)}{  \sqrt{Q^2(Q^2 + Q^2_t)}  }
\end{equation}

\begin{figure}[h]
    \includegraphics[width = 0.8 \textwidth]{figures/cost_st_vs_cost.pdf}
    \centering
    \caption{A comparison of the true angular distribution (given by $c_*$) and the reconstructed
        one, based on guessing the lepton pair boost direction to be the incident quark direction ($c_r$)
    from MC simulation.}
    \label{fig:cost_st}
\end{figure}

where $Q$ represents the four momentum of the lepton pair, $Q_t$ represents the transverse momentum of 
the lepton pair, $l$ ($\bar l$) represents the four momentum of $\ell^-$ ($\ell ^+$) with 
$l^{\pm} = \frac{1}{\sqrt{2}}(l_E \pm l_z)$. Crucially the sign of $cos \theta^{*}_{CS}$ depends 
on the direction of the incident quark. 

In a proton-proton collider like the LHC, the quark and anti-quark directions for each collision are 
not known. However, on average the quark will carry a larger momentum fraction than the anti-quark because
anti-quark must come from the parton sea, and the quark is likely to be a valence quark. Thus, usually the
lepton pair will be boosted along the direction of the incident quark. 
When reconstructing $cos \theta^{*}_{CS}$ we can therefore take the direction of the lepton pair along the beam axis
to be the likely direction of the incident quark. 
The effect of incorrectly guessing the quark direction is to dilute the true asymmetry.
Figure \ref{fig:cost_st} shows the effect of this dilution on the angular distribution.
It is more likely that the lepton pair direction matches the quark direction
when the lepton pair has a larger boost (ie a larger $x_F$) and becomes equally probable as improbable 
as $x_F \rightarrow 0$.  Figure \ref{fig:dilu} demonstrates that this dilution effect
becomes small at large $x_F$. \\


\begin{figure}[H]
    \begin{subfigure}{0.5 \textwidth}
        \includegraphics[width = 0.9 \textwidth] {figures/quark_guessing.pdf}
    \end{subfigure}
    \begin{subfigure}{0.5 \textwidth}
        \includegraphics[width = 0.9 \textwidth] {figures/dilution.pdf}
    \end{subfigure}
    \caption{On the left, a comparison of correct and incorrect signing of $cos \theta^*_{CS}$ 
    as a function of $x_F$ based on MC studies.
    One the right, the dilution factor, $D = (N_c - N_i)/ (N_c + N_i)$, plotted as
    a function of Feynman x, note that it becomes large at large $x_F$.
}
\label{fig:dilu}
\end{figure}


\section{Analysis Scheme}

In the CMS detector, it is straight forward to identify and reconstruct muons, we therefore restrict this analysis 
to muon pairs, and leave a similar analysis on electron pairs for future work. 
We can reconstruct the three key variables $x_\mathrm{r}$, $M_\mathrm{r}$, and $c_\mathrm{r}$ from the muon information, 
with direction of the pair along the beam axis can be taken as the likely quark direction for the $q\bar q$ initial state.  
The distribution function of the reconstructed variables can be expressed as a convolution of the cross section defined in equation~\ref{eq:tot_xsec} 
(with the $q\bar q$ cross section given by equation~\ref{eq:qqbar_xsec}),

\begin{equation}
    f(x_\mathrm{r},M_\mathrm{r},c_\mathrm{r}) = C \int dx_\mathrm{F}dMdc_* R(x_\mathrm{r},M_\mathrm{r},c_\mathrm{r}; x_\mathrm{F}, M, c_*)
    \varepsilon (x_\mathrm{F}, M, c_*) \frac{d^3\sigma}{dx_\mathrm{F}dM dc_*} 
\end{equation}

where $C$ is a normalization constant, $R$ is a ``resolution function'' that incorporates real detector resolution and parton shower effects, 
and $\varepsilon$ is an efficiency function.  
The key point is that the linearity of the $c_*$-odd term in equation~\ref{eq:qqbar_xsec} is not disturbed by the convolution and the linear coefficient $A_{FB}^{(1)}$ is unaffected.  
The linearity of the problem also allows the fitting function to be represented by a set of parameter-independent 3D histograms or templates.  
The DY templates can be constructed by re-weighting MC simulated events based on generator-level information to calculate the weights, and binning in 
the reconstructed variables. 
The background distributions $f^j_\mathrm{bk}(x_\mathrm{r},M_\mathrm{r},c_\mathrm{r})$ can be extracted directly from 
fully simulated samples by binning in the reconstructed variables.  
The various parts of the $q\bar q$ distribution can be constructed by re-weighting 
simulated data using generator-level variables to generate the weights and binning in reconstructed variables. 

To illustrate the re-weighting procedure, let's assume that we have a sample of fully simulated and reconstructed $q\bar q \to \ell^+\ell^-$ events.  
The events will have $q\bar q$, $qg$, and $\bar qg$ initial states.  
We use all of them but need to take care when gluons [virtual $q\bar q$ pairs] are present.  
The simulation generates a $c_*$ distribution that is asymmetric, but we use each event twice to generate a symmetric template 
and we use it twice to generate an anti-symmetric template.  
Each event enters the symmetric template $f_\mathrm{s}(x_\mathrm{r}, M_\mathrm{r}, c_\mathrm{r})$ twice at $\pm c_\mathrm{r}$ 
with unit weight and normalize it by the total number of entries [twice the total number of events].  
We generate the asymmetric distribution by applying the following weight to each simulated event using generator-level quantities and our fit to $\alpha$,

\begin{equation}
    w_\mathrm{a}(M^2, c_*) = \frac{4}{3}\cdot\frac{2+\alpha}{1+c_*^2+\alpha\left(1-c_*^2\right)}c_*
\end{equation}

and then binning the weighted events in the reconstructed quantities to produce the asymmetric distribution $f_\mathrm{a}(x_\mathrm{r}, M_\mathrm{r}, c_\mathrm{r})$ 
with the same normalization as used for the symmetric distribution.  
Note that $c_*$ and $c_\mathrm{r}$ are different quantities and can have different signs.  
The sign of $c_*$ follows from the quark direction and the $\ell^-$ direction.  
We start by assuming that the quark is moving in the $+z$ direction to calculate $c_*$.  
If $p_z$ of initial state $q$ is negative, the sign of $c_*$ is flipped.  Note that in $g\bar q$ events, 
the flip is done if the $p_z$ of the $\bar q$ is positive.  
This event weight is applied to $c_\mathrm{r}$ bin that is signed independently using $p_z$ of the $\ell^+\ell^-$ pair.  
The flip is performed if $p_z<0$.  The anti-symmetrization is performed by flipping the sign of the weight and the bin in $c_\mathrm{r}$.  
Then a simple two parameter likelihood fit to the real data would follow from the following histograms,

\begin{align}
    f(x_\mathrm{r},M_\mathrm{r},c_\mathrm{r}) = R_\mathrm{bk} \sum_jf^j_\mathrm{bk}(x_\mathrm{r},M_\mathrm{r},c_\mathrm{r})+
    \biggl(1 - R_\mathrm{bk}\biggr )   \left[f_\mathrm{s}(x_\mathrm{r}, M_\mathrm{r}, c_\mathrm{r})+
    A_\mathrm{FB}f_\mathrm{a}(x_\mathrm{r}, M_\mathrm{r}, c_\mathrm{r})\right]
    \label{eq:template_schemeone}
\end{align}

where the background fraction $R_\mathrm{bk}$ and asymmetry $A_\mathrm{FB}$ are allowed to float.  
Note that the backgrounds can be summed into a single distribution and represented by a single parameter, 
as long as the relative normalization of each background sample is correct.  
This analysis should be done in bins or slices of $M_\mathrm{r}$ so that it is really a series of fits and extracts $A_\mathrm{FB}(M)$.  
Note that this technique automatically accounts for resolution, dilution, migration, and acceptance effects so long as 
they are correctly modeled in the simulation. \\


A small fraction of the Drell-Yan events will come from initial partons with no asymmetry, (ie $gg, \bar{q}\bar{q}, qq$).
Such events will still have a $q\bar{q}$ annihilation, but we have no hope of ascertaining the initial quark direction, 
and thus no way to recovery the asymmetry. We therefore do not include these events as a part of our signal templates, but
rather as a separate background template. 
According to MC studies, we expect these type of events to comprise only 3\% of the total DY events. \\

Drell-Yan events producing a pair of tau leptons which then each decay to muons would provide a background 
to our signal. Due to the two neutrinos emitted in the tau decay, these events would have a large Missing Transverse Energy,
(MET) and the muon pair would typically have a lower invariant mass than the original tau pair. The tau pair would of course have the
same Drell-Yan asymmetry, and in a high energy collision the boost of the tau lepton would cause the produced muon to retain the tau's direction.
Thus reconstructing the final muon pair would retain part of the tau pair's asymmetry. Because of the unlikely double muon decay and
additional suppression from the MET selection requirement (discussed later) these events comprise and extremely small part of of our 
selected events (less than 1\%).  
Yet in order to avoid having a background template with an asymmetry similar to our signal, we include such tau events as part of our signal
templates, binning in the reconstructed muon variables but weighting based on the true tau variables. Figure \ref{fig:tautau_cost} 
shows reconstructed and true angular distribution of these $\tau\tau$ events.


\begin{figure}[H]
    \includegraphics[width = 0.5 \textwidth]{figures/tautau_cost.pdf}
    \centering
    \caption{A comparison of the true angular distribution (given by $c_*$) of the tau tau pair
    and the angular distribution of the reconstructed muon pair (given by $c_r$). } 
    \label{fig:tautau_cost}
\end{figure}

%----------------------------------------------------------------------------------------
%	Data and MC Samples
%----------------------------------------------------------------------------------------
\section{Data and MC Samples}

Our measurements of the forward-backward asymmetry are based on data recorded in 2016 from the CMS detector, at $\sqrt{s} =$ 13 TeV
corresponding to 35.87 fb$^{-1}$. Data corresponds to ``03Feb2017ReReco`` reprocessing and uses the
{\small Cert\_271036-284044\_13TeV\_23Sep2016ReReco\_Collisions16\_JSON.txt} file to select the luminosity sections.
Table \ref{tab:datasets} summarizes the data samples used in this analysis.\\

Monte Carlo samples used in this analysis have been created from a variety of generators and processed with Pythia8 and the full CMS detector simulation.
The Drell-Yan samples have been generated at NLO with aMC@NLO. Powheg has been used for the $t \bar{t}$, $tW$ and $\bar{t}W$ samples. 

Table \ref{tab:mcsets} summarizes the Monte Carlo Samples used in this analysis.\\

\begin{table}[H]
    \centering

    \caption{ Datasets used in this analysis. Data corresponds to Feb 3 ``Re-MiniAOD'' and uses 
        {\small Cert\_271036-284044\_13TeV\_23Sep2016ReReco\_Collisions16\_JSON.txt} 
        file to select the luminosity sections.
        Corresponding to a total integrated luminosity of 35.9 fb$^{-1}$}
        \label{tab:datasets}
        \begin{tabular}{|| l c ||}
            \hline
            \textbf{Singe Muon DataSet}  & \textbf{Luminosity (fb$^{-1}$)}\\
            \hline \hline

%
            SingleMuon\_Run2016B\_03Feb2017\_v2 &   5.75  \\
            SingleMuon\_Run2016C\_03Feb2017   &  2.57 \\
            SingleMuon\_Run2016D\_03Feb2017   & 4.24 \\
            SingleMuon\_Run2016E\_03Feb2017   & 4.02 \\
            SingleMuon\_Run2016F\_03Feb2017   & 3.10 \\
            SingleMuon\_Run2016G\_03Feb2017   & 7.58 \\
            SingleMuon\_Run2016H\_03Feb2017\_v2 & 8.43 \\
            SingleMuon\_Run2016H\_03Feb2017\_v3&  0.22 \\



            \hline

        \end{tabular}

\end{table}

\begin{table}[H]
        \centering
        \caption{ Monte Carlo Samples used in this analysis. }
        \label{tab:mcsets}
        \begin{adjustbox}{width=1\textwidth}
            \begin{tabular}{||l l l ||} 
                \hline
                \textbf{Process} & \textbf{MCs Dataset} & \textbf{Cross Section (pb)} \\ [0.5ex] 
                \hline\hline
                $DY \rightarrow l^+ l^-$ & {\small DYJetsToLL\_M-100to200\_TuneCUETP8M1\_13TeV-amcatnloFXFX-pythia8} & 226 \\ 
                & {\small DYJetsToLL\_M-200to400\_TuneCUETP8M1\_13TeV-amcatnloFXFX-pythia8}   &   7.67   \\  
                & {\small DYJetsToLL\_M-400to500\_TuneCUETP8M1\_13TeV-amcatnloFXFX-pythia8}    &   0.423  \\
                & {\small DYJetsToLL\_M-500to700\_TuneCUETP8M1\_13TeV-amcatnloFXFX-pythia8}    &   0.24   \\ 
                & {\small DYJetsToLL\_M-700to800\_TuneCUETP8M1\_13TeV-amcatnloFXFX-pythia8}    &   0.035 \\ 
                & {\small DYJetsToLL\_M-800to1000\_TuneCUETP8M1\_13TeV-amcatnloFXFX-pythia8}   &   0.03  \\ 
                & {\small DYJetsToLL\_M-1000to1500\_TuneCUETP8M1\_13TeV-amcatnloFXFX-pythia8}  &   0.016 \\ 
                & {\small DYJetsToLL\_M-1500to2000\_TuneCUETP8M1\_13TeV-amcatnloFXFX-pythia8}  &   0.002 \\ 
                & {\small DYJetsToLL\_M-2000to3000\_TuneCUETP8M1\_13TeV-amcatnloFXFX-pythia8}  &   0.00054\\
                \hline
                $t \bar{t}$              & TT\_TuneCUETP8M2T4\_13TeV-powheg-pythia8  & 831.76\\ 
                \hline
                $tW$                     & ST\_tW\_antitop\_5f\_inclusiveDecays\_13TeV-powheg-pythia8\_TuneCUETP8M1   & 35.85\\
                $\bar{t}W$               & ST\_tW\_antitop\_5f\_inclusiveDecays\_13TeV-powheg-pythia8\_TuneCUETP8M1   & 35.85\\
                \hline
                $WW$                     & WW\_TuneCUETP8M1\_13TeV-pythia8 & 63.21 \\
                $WZ$                     & WZ\_TuneCUETP8M1\_13TeV-pythia8 & 47.13 \\
                $ZZ$                     & ZZ\_TuneCUETP8M1\_13TeV-pythia8 & 16.523 \\ [1ex]

                \hline \hline
            \end{tabular}
        \end{adjustbox}
    \end{table}



%-------------------------------------------------------------------------------------
%------- EVENT SELECTION
%--------------------------------------------------------------------------------
    \section{Event Selection}

    This analysis is based on Particle Flow (PF) objects. All data and MC events are subject to the following criteria.

    The first step in event selection is the trigger. The trigger used is an OR of two single muon triggers, HLT\_IsoMu24 and HLT\_IsoTkMu24. 
    Events are required to have a leading muon with $p_t > 26$ GeV and a subleading muon with $p_t > 10$ GeV. The muons are both required 
    to have $|\eta| < 2.4$ and must be oppositely charged. \\

    Muons are also required to pass the additional muon-quality requirements of the ''tight'' selection 
    criteria set forth by the Muon POG (INSERT REF HERE). These requirements include, among others: \\

    \begin{itemize}
        \item The candidate is reconstructed as a Global Muon. 
        \item Its tracker track has transverse impact parameter $d_{xy} < 2 $ mm . 
            This is done in order to reduce background from cosmics and decays in flight. 
        \item The longitudinal distance of the tracker track wrt. the primary vertex is $d_{z} < 5 $ mm.
            Done in order to further reduce background from cosmics and decays in flight.
        \item Number of tracker layers with hits $> 5$. This is done to guarantee a good p$_{t}$ measurement.
    \end{itemize}

    An additional isolation requirement is applied to muons, based on the muon POG's recommendation. 
    It is a tight Particle-Flow based combined relative isolation requirement with $\Delta \beta$ corrections.
    Muons are required to be isolated from tracks within a cone size of $\Delta R = 0.4$, with 
    $\Delta R = \sqrt{(\Delta \eta)^{2} + (\Delta \phi)^2}$\\

    In order to suppress $t \bar{t}$ backgrounds additional selection criteria are applied. 
    Missing transverse energy (MET) is defined as the energy which was not detected by the CMS 
    detector, but must exist in order to satisfy conservation of energy and momentum. 
    It is computed as the negative sum of all PF candidates' transverse 
    energies. The MET used in this analysis is corrected to take into account Jet Energy Corrections
    (JECs) and in simulation it also corrected for the adjust jet energy resolution (INSERT REF HERE). 
    Events are required to have MET $ < 50 $ GeV.

    There is an anti-B-tag requirement based on jets from the Anti-kT 
    algorithm with a cone size of 0.4 (INSERT REF HERE). 
    The anti-B-tag implies the jet is not likely to have come from a b quark, based on the Combined MultiVAriate Algorithm (CMVA) (INSERT REF HERE).
    Failing a B-tag is equivalent to the CMVA discriminator variable for that jet being less than the B-tag POG's medium working point of 0.4432 (INSERT REF HERE).
    The selection requirement is that the two leading jets in the event must fail a B-tag. 
    Going through the list of jets, ordered by P$_t$, the first 2 jets with $|\eta| < 2.4$ are selected and required to fail a B-tag. \\

    The selection criteria are summarized in Table \ref{tab:cuts}. \\


    \begin{table}[H]
        \centering

        \caption{ Selection criteria used in this analysis.}
        \label{tab:cuts}
        \begin{adjustbox}{width=1\textwidth}
            \begin{tabular}{|| c | l ||}
                \hline
                \textbf{Selection Cut}  & \textbf{Cut Threshold}\\
                \hline \hline
                Trigger & HLT\_IsoMu24 OR HLT\_IsoTkMu24 \\

                \hline

                Kinematic Acceptance & A leading muon with $p_T > 26 $ GeV 
                and subleading muon with $p_T > 10 $ GeV.\\
                & Both muons with $|\eta| < 2.4$ and opposite charge. \\
                \hline
                Muon Identification & Tight Muon \\

                \hline

                Muon Isolation & $\sum p_t (\text{Charged Hadrons}) +$  \\
                & $max(0, \sum E_t (\text{Neutral Hadrons}) + \sum E_t (\text{photons}) - $ \\ 
                &  $0.5 * \sum p_t (\text{Charged hadrons from Pileup}))\; /p_t (\mu) \; < 0.15$ \\

                \hline

                $t\bar{t}$ Background Rejection & $MET < 50$ GeV \\
                & $CMVA < 0.4432$ for two highest p$_t$ jets with $|\eta| < 2.4$.  \\




                \hline

            \end{tabular}
        \end{adjustbox}

    \end{table}

    \section{Efficiency Corrections}
    Some corrections must be made to account for the differences between data and simulation. These corrections are discussed here.
    Generally, scalefactors, equal to efficiency in data divided by the efficiency in MC, are applied to all simulated events. \\

    \subsection{Muon Trigger and Identification Efficiencies}
    The efficiency of the single muon triggers combination used in this analysis (HLT\_IsoMu24 or HLT\_IsoTkMu24) is not exactly the same in data and simulation. 
    The muon selection requirement, IsTightMuon, has different efficiencies between data and simulation.
    Additionally the muon isolation requirement has different efficiencies between data and simulation.
    These efficiencies have been studied by the muon POG (INSERT REF HERE).
    The scale factors, binned p$_t$ and $|\eta|$, have been provided.
    For each event, there is one scalefactor corresponding to the trigger efficiency based on the muon that is matched to the trigger,
    two scale factors for the selection requirement for each muon, and two scale factors for the isolation requirement for each muon. 
    The scalefactors are different for Run2016B/C/D/E/F data and Run2016G/H data, therefore a combined scalefactor based on the weighted average of 
    the two different scalefactors, weighted by the integrated luminosities of the corresponding runs is the one actually applied. 



    \subsection{Anti-b-tag Efficiency}
    There are differing efficiencies in data and MC for the anti-b-tag requirement. Scalefactors for passing 
    a b-tag, binned in jet p$_t$ and $\eta$ for different flavors of jets ($f = b, c, udsg$) have been provided 
    by the b-tag POG (INSERT REF HERE). MC efficiencies for passing a b-tag, for different flavors of jets,
    were computed from simulation as:

    \begin{equation}
        \varepsilon_f(p_t, |\eta|) = \frac{N_f^{b-tagged}(p_t, |\eta|)}{N_f^{total}(p_t, |\eta|)}
        \nonumber
    \end{equation}

    Simulated events are then re-weighted with a factor: \\
    \begin{center}
        \resizebox{0.45 \textwidth}{!}{
            $
            w = \prod\limits_{i=1,2} \frac{1 - SF_f(p_{t_i}, |\eta_i|) \; \varepsilon_f (p_{t_i}, |\eta_i|)} {1 - \varepsilon_f (p_{t_i}, |\eta_i|)}
            $
        }
    \end{center}
    where the product is over the two jets failing the CMVA medium working point. \\

    \begin{figure}[P]
        \centering
        \includegraphics[width = 0.9 \textwidth]{figures/m_data_vs_mc.pdf}
        \caption{A comparison of data and MC samples after all SF have been applied and 
            $e \mu$ scaling has been done. Note that there are still an excess of data events up until approximately
            300 GeV.
        }
        \label{fig:m_data_vs_mc}
    \end{figure}

    \begin{figure}[P]
        \centering
        \includegraphics[width = 0.9 \textwidth]{figures/cost_data_vs_mc.pdf}
        \caption{A comparison of the data and MC reconstructed angular distributions. }
        \label{fig:cost_data_vs_mc}
    \end{figure}


    \subsection{Pileup Re-Weigthing}
    Needs to be done. \\

    \section{Background Estimation with $e \mu$}
    It is preferable to estimate our background from data, rather than relying solely on MC. 
    Because all of our non-QCD background muon pairs come from particles which also have 
    $e \mu$ decay channels, we can estimate our backgrounds with the $e \mu$ method (INSERT REF HERE). 

    The expected ratio of $e\mu$ and $\mu \mu$ events is 2:1, but differing acceptances and efficiencies can 
    cause deviations. For each background source can estimate the number of muon pairs from:

    \begin{equation}
        N^{\mu \mu}_{est} = \frac{N^{\mu \mu}_{MC}}{N^{e\mu}_{MC}} \; N^{e\mu}_{obs}
                      = \frac{N^{\mu \mu}_{MC}}{N^{e\mu}_{MC}} \: 
                        \frac{N^{e\mu}_{MC}}{N^{e\mu}_{total \; MC}} N^{e \mu}_{data}
        = \frac{N^{e \mu}_{data}}{N^{e\mu}_{total \: MC}} N^{e\mu}_{MC}
        \label{eq:emu}
    \end{equation}

    Thus we can simply scale each background source by the ratio of observed and MC $e \mu$ events. \\

    To select our $e \mu$ events we apply a very similar selection criteria to our $\mu \mu$ events. 
    In addition to the 'or' of the two muon triggers 
    we also add an electron trigger, HLT\_Ele23\_WPLoose\_Gsf.
    We still require the leading lepton to have p$_t$ > 26 GeV and the subleading lepton
    to have p$_t$ > 10 GeV. The muon identification criteria are the same.
    For electron identification, we use the Multivariate Electron Identification, and require 
    electrons to pass the 80\% signal efficiency working point (INSERT REF HERE).
    We still require the leptons to have opposite charge. 
    We apply the same MET and anti-b-tagging selection requirements. 
    In order to match efficiencies between data and MC we apply the same muon and b-tagging scale 
    factors and additional scalefactors for the electron identification and trigger (INSERT REF HERE)
    (Note: electron HLT SF not yet applied).
    Because the mass of the electron and muon are both negligible at this energy, we expect the
    invariant mass distribution of these $e \mu$ events to match that of our signal $\mu \mu$ events.
    We therefore require events to have the invariant mass of the lepton pair, $M_{e\mu}$ > 150 GeV, 
    to match the kinematic region of our analysis. 
    There are small amount of $DY \rightarrow \tau \tau$ events which decay to an $e \mu$ pair,
    because we don't include $\tau \tau$ events in our background, we subtract the number of MC 
    $\tau \tau$ events from the data events before computing the ratio of data to MC. \\

    The overall ratio of data to MC events is 1.05 $\pm$ 0.01. This ratio is used to scale
    background samples. 
    Figure \ref{fig:emu} shows a comparison between data and MC $e \mu$ events.
    Figures \ref{fig:m_data_vs_mc} and \ref{fig:cost_data_vs_mc} show a comparison 
    between MC samples and data after this $e \mu$ scaling. 


    \begin{figure}[H]
        \centering
        \includegraphics[width = 0.9 \textwidth]{figures/emu_m.pdf}
        \caption{A comparison of observed $e \mu$ events and MC $e \mu$ events. The ratio
        appears to have some dependence on mass, possibly due to contamination from QCD jets.}
        \label{fig:emu}
    \end{figure}
    




    \newpage
    \section{Results}

    The results for the fits in different mass bins are presented in Table \ref{tab:results} and shown
    graphically in Figure \ref{fig:AFB}. There appears to be a dip in AFB in the mass range 200 to 250 GeV.\\

    As a cross-check, the fitted background fraction is compared to the background fraction expected from 
    MC samples is shown in Figure \ref{fig:back_frac}.
    In the lowest three mass bins the fitted background fraction is significantly higher than the MC
    prediction. This discrepancy needs to be addressed. \\

    \begin{table}[h]
        \centering

        \caption{ Results of fit in different mass bins. A$_{FB}$ is the forward-backward asymmetry.
        $R_\mathrm{bk}$ is the fraction of background events in the sample. }
        \label{tab:results}
        \begin{tabular}{|| c | c | c | c | c ||}
            \hline
            Mass Bin (GeV)  & Number of Events &  
            A$_{FB}$ & $R_\mathrm{bk}$\\
            \hline \hline
            150-200 & 77937 & $0.65 \pm 0.02$ & $0.195 \pm 0.007$    \\
            200-250 & 25572 & $0.56 \pm 0.03$ & $0.163 \pm 0.011$   \\
            250-350 & 16269 & $0.62 \pm 0.03$ & $0.198 \pm 0.011$   \\
            350-500 & 5698 & $0.63 \pm 0.04$ & $0.168 \pm 0.014$    \\
            500-700 & 1557 & $0.58 \pm 0.06$ & $0.152 \pm 0.025$    \\
            $\geq$ 700 & 471 & $0.59 \pm 0.09$ & $0.100 \pm 0.030$   \\



            \hline

        \end{tabular}

    \end{table}

    \section{Further Work}
    To explain the background fraction in the first bin as well the low asymmetry in the 2nd mass bin, further work needs to be done.
    The most obvious issue is that the MC samples used to generate the signal templates for the first two mass regions have quite poor statistics. 
    The ratio of data events to MC events is approximately 6 to 1 in the first bin and 1 to 1 in the 2nd bin. This is made even worse by the fact that some
    of the MC events have negative generator weights. This causes data events to hit bins with few or zero MC events, or bins with only negatively weighted
    MC events, which makes the fit results unreliable. \\

    Pileup re-weighting needs to be done but we don't expect this to be a large effect. \\

    There is an additional background from QCD jets misidentified as muons. Based on (INSERT 13 TeV DY REF) we expect this to contribute at
    about the 3\% level for masses in the range 150 to 250 GeV, which would help explain the discrepancy seen in 
    Figures \ref{fig:m_data_vs_mc} and \ref{fig:cost_data_vs_mc}. \\







    \begin{figure}[h]
        \includegraphics[width = 0.9 \textwidth]{figures/AFB.pdf}
        \caption{Drell-Yan forward-backward asymmetry as a function of mass}
        \label{fig:AFB}
    \end{figure}

    \begin{figure}[h]
        \includegraphics[width = 0.9 \textwidth]{figures/back_frac.pdf}
        \caption{The fitted background fraction (black) as compared to the 
            background fraction expected based on MC samples (purple). 
            All backgrounds have been scaled with the ratio of data and MC $ e \mu$ events.
            The background fraction contributions for each type of background are shown for comparison. 
            }
        \label{fig:back_frac}
    \end{figure}


%----------------------------------------------------------------------------------------


    \bibliographystyle{plain}
    \bibliography{Drell_Yan_asymmetry}


    \end{document}
