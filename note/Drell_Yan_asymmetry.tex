
%----------------------------------------------------------------------------------------
%	PACKAGES AND OTHER DOCUMENT CONFIGURATIONS
%----------------------------------------------------------------------------------------

\documentclass[
10pt, % Main document font size
a4paper, % Paper type, use 'letterpaper' for US Letter paper
oneside, % One page layout (no page indentation)
%twoside, % Two page layout (page indentation for binding and different headers)
headinclude,footinclude, % Extra spacing for the header and footer
BCOR5mm, % Binding correction
]{scrartcl}

\usepackage{adjustbox}
\usepackage{amsmath}
\usepackage{graphicx}
\graphicspath{{figures/}}

\usepackage[margin=0.5in]{geometry}

%\input{structure.tex} % Include the structure.tex file which specified the document structure and layout

\hyphenation{Fortran hy-phen-ation} % Specify custom hyphenation points in words with dashes where you would like hyphenation to occur, or alternatively, don't put any dashes in a word to stop hyphenation altogether

%----------------------------------------------------------------------------------------
%	TITLE AND AUTHOR(S)
%----------------------------------------------------------------------------------------

\title{Meauring DY Asymmetry} % The article title

\author{Oz Amram \& Morris Swartz\textsuperscript{1}} % The article author(s) - author affiliations need to be specified in the AUTHOR AFFILIATIONS block

\date{} % An optional date to appear under the author(s)

%----------------------------------------------------------------------------------------

\begin{document}


%----------------------------------------------------------------------------------------
%	TABLE OF CONTENTS & LISTS OF FIGURES AND TABLES
%----------------------------------------------------------------------------------------

\maketitle % Print the title/author/date block

\setcounter{tocdepth}{2} % Set the depth of the table of contents to show sections and subsections only

\tableofcontents % Print the table of contents


%----------------------------------------------------------------------------------------
%	ABSTRACT
%----------------------------------------------------------------------------------------

\section*{Abstract} % This section will not appear in the table of contents due to the star (\section*)
Placeholder


%----------------------------------------------------------------------------------------
%	AUTHOR AFFILIATIONS
%----------------------------------------------------------------------------------------

{\let\thefootnote\relax\footnotetext{1 \textit{Department of Physics and Astronomy, Johns Hopkins University, }}}


%----------------------------------------------------------------------------------------

\newpage % Start the article content on the second page, remove this if you have a longer abstract that goes onto the second page

%----------------------------------------------------------------------------------------
%	INTRODUCTION
%----------------------------------------------------------------------------------------

\section{Introduction}

Placeholder

 
%----------------------------------------------------------------------------------------
%	Data and MC Samples
%----------------------------------------------------------------------------------------

\section{Data and MC Samples}

Our measurements of the forward-backward asymmetry are based on data recorded in 2016 from the CMS detector, at $\sqrt{s} =$ 13 TeV
corresponding to 35.87 fb$^{-1}$. Data corresponds to ``23SepReReco`` reprocessing and uses the
{\small Cert\_271036-284044\_13TeV\_23Sep2016ReReco\_Collisions16\_JSON.txt} file to select the luminosity sections.
Table \ref{tab:datasets} summarizes the data samples used in this analysis.\\

Monte Carlo samples used in this analysis have all been generated with PYTHIA8 and processed with the full CMS detector simulation.
Table \ref{tab:mcsets} summarizes the Monte Carlo Samples used in this anaysis.\\


\begin{table}[htp]
\centering
\label{tab:datasets}
    \caption{ Datasets used in this analysis. Data corresponds to ``23SepReReco'' reprocessing and uses 
{\small Cert\_271036-284044\_13TeV\_23Sep2016ReReco\_Collisions16\_JSON.txt} file to select the luminosity sections.}
\begin{adjustbox}{width=1\textwidth}
 \begin{tabular}{||c c c ||} 
 \hline
 \textbf{Process} & \textbf{MC Dataset} & \textbf{Cross Section (pb)} \\ [0.5ex] 
 \hline\hline
 $DY \rightarrow l^+ l^-$ & {\small DYJetsToLL\_M-100to200\_TuneCUETP8M1\_13TeV-amcatnloFXFX-pythia8} & 226 \\ 
                          & {\small DYJetsToLL\_M-200to400\_TuneCUETP8M1\_13TeV-amcatnloFXFX-pythia8}   &   7.67   \\  
                          & {\small DYJetsToLL\_M-400to500\_TuneCUETP8M1\_13TeV-amcatnloFXFX-pythia8}    &   0.423  \\
                          & {\small DYJetsToLL\_M-500to700\_TuneCUETP8M1\_13TeV-amcatnloFXFX-pythia8}    &   0.24   \\ 
                          & {\small DYJetsToLL\_M-700to800\_TuneCUETP8M1\_13TeV-amcatnloFXFX-pythia8}    &   0.035 \\ 
                          & {\small DYJetsToLL\_M-800to1000\_TuneCUETP8M1\_13TeV-amcatnloFXFX-pythia8}   &   0.03  \\ 
                          & {\small DYJetsToLL\_M-1000to1500\_TuneCUETP8M1\_13TeV-amcatnloFXFX-pythia8}  &   0.016 \\ 
                          & {\small DYJetsToLL\_M-1500to2000\_TuneCUETP8M1\_13TeV-amcatnloFXFX-pythia8}  &   0.002 \\ 
                          & {\small DYJetsToLL\_M-2000to3000\_TuneCUETP8M1\_13TeV-amcatnloFXFX-pythia8}  &   0.00054\\
 \hline
 $t \bar{t}$ & TT\_TuneCUETP8M2T4\_13TeV-powheg-pythia8  & 831.76\\ [1ex] 
 \hline \hline
\end{tabular}
\end{adjustbox}
\end{table}


\begin{table}[htp]
    \centering
    \label{tab:mcsets}

 \caption{ Monte Carlo Samples used in this analyis.}
\begin{tabular}{|| c c ||}
    \hline
    \textbf{Singe Muon DataSet}  & \textbf{Luminosity (fb$^{-1}$)}\\
 \hline \hline

  /SingleMuon/Run2016B-23Sep2016-v3/MINIAOD   &   {\small 5.8}  \\
  /SingleMuon/Run2016C-23Sep2016-v1/MINIAOD    &  {\small 2.5}  \\
  /SingleMuon/Run2016D-23Sep2016-v1/MINIAOD    &  {\small 4.3}  \\ 
  /SingleMuon/Run2016E-23Sep2016-v1/MINIAOD    &  {\small 4.1}  \\
  /SingleMuon/Run2016F-23Sep2016-v1/MINIAOD    &  {\small 3.1}  \\
  /SingleMuon/Run2016G-23Sep2016-v1/MINIAOD    &  {\small 7.5}  \\
  /SingleMuon/Run2016H-PromptReco-v3/MINIAOD   &  {\small 8.5}  \\  



 \hline

 \end{tabular}

\end{table}

%-------------------------------------------------------------------------------------
%------- EVENT SELECTION
%--------------------------------------------------------------------------------
\section{Event Selection}

This analysis is based on particle flow objects. All data events are subject to the following criteria.

The first step in event selection is the trigger. The trigger used is an OR of two single muon triggers, HLT\_IsoMu24 and HLT\_IsoTkMu24. 
Events are required to have a leading muon with $p_T > 26 GeV$ and a subleading muon with $p_T > 10 GeV$. The muons are both required 
to have $|\eta| < 2.4$ and must be oppositely charged. 
Muons are also required to pass the additional muon-quality requirements of the ''tight'' selection 
criteria set forth by the Muon POG (INSERT REF HERE). \\

An additional isolation requirement is applied to muons. It is a tight Particle-Flow based combined relative isolation requirement with $\Delta \beta$ corrections. \\

In order to surpress $t \bar{t}$ backgrounds additional selection criteria are applied. 
Events are required to have missing transverse momentum (MET) $ < 50 GeV$.
There is an anti-b tag requirement based on jets from the Anti-kT 
Pileup Per Particle Idenitification (PUPPI) algorithm with a cone size of 0.4 (INSERT REF HERE). 
The requirement is based on the Combined MultiVAriate Algorithm (CMVA) (INSERT REF HERE) and requires that the CMVA discriminator variable for each jet 
be less than 0. \\

The selection criteria are summarized in Table \ref{tab:cuts}. \\


\begin{table}[htp]
    \centering
    \label{tab:cuts}

 \caption{ Selection criteria used in this analysis.}
\begin{adjustbox}{width=1\textwidth}
\begin{tabular}{|| c | c ||}
    \hline
    \textbf{Selection Cut}  & \textbf{Cut Threshold}\\
 \hline \hline
 Trigger & HLT\_IsoMu24 OR HLT\_IsoTkMu24 \\

 \hline

 Kinematic Acceptance & A leading muon with $p_T > 26 GeV$ and sublead muon with $p_T > 10 GeV$.\\
                      & Both muons with $|\eta| < 2.4$ and opposite charge. \\
  \hline
  Muon Identification & Tight Muon \\

  \hline

  Isolation & $\sum p_T (\text{Charged Hadrons}) +$  \\
            & $max(0, \sum E_T (\text{Neutral Hadrons}) + \sum E_T (\text{photons}) - 
              0.5 * \sum p_T (\text{Charged hadrons from Pileup}))/p_T (\mu) < 0.15$ \\

  \hline

  $t\bar{t}$ Background Rejection & $MET < 50 GeV$ and $CMVA < 0$ for both jets. \\




 \hline

 \end{tabular}
 \end{adjustbox}

\end{table}

%------------------------------------------------------------------------------
% ------------ Results 
%  --------------------------------------------------------------------------

\section{Results}

The results for the fits in different mass bins are presented in Table \ref{tab:results}.


\begin{table}[htp]
    \centering
    \label{tab:results}

 \caption{ Results of fit in different mass bins. AFB is the forward-backward asymmetry.
 $R_{t\bar{t}}$ is the fraction of $t\bar{t}$ events in the sample.
 $R_{\text{noasym}}$ is the fraction of Drell-Yan events coming from production processes with no asymmetry 
 ($gg, qq, \bar{q}\bar{q}$).}
\begin{tabular}{|| c | c | c | c | c ||}
    \hline
    Mass Bin (GeV)  & Number of Events &  
    AFB & $R_{t \bar{t}}$ & $R_{\text{noasym}}$  \\
 \hline \hline
 150-200 & 56390 & $0.61 \pm 0.02$ & $0.122 \pm 0.007$ & $0.0000 \pm 0.0008$  \\
 200-250 & 18616 & $0.56 \pm 0.03$ & $0.125 \pm 0.011$ & $0.0085 \pm 0.0053$ \\
 250-350 & 11884 & $0.64 \pm 0.03$ & $0.158 \pm 0.011$ & $0.0032 \pm 0.0041$ \\
 350-500 & 4196 & $0.63 \pm 0.05$ & $0.154 \pm 0.015$  & $0.0059 \pm 0.0130$ \\
 500-700 & 1121 & $0.61 \pm 0.06$ & $0.142 \pm 0.024$  & $0.0000 \pm 0.0130$ \\
 $\geq$ 700 & 329 & $0.58 \pm 0.11$ & $0.099 \pm 0.034$ & $0.0000 \pm 0.0151$ \\



 \hline

 \end{tabular}

\end{table}

%\begin{figure}[tb]
%    \includegraphics{Afb_vs_mass}
%    \caption{Assymmetry as a function of mass}
%\end{figure}
    


%----------------------------------------------------------------------------------------

\end{document}
